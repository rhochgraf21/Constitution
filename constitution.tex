% BEFORE CHANGES ARE MADE TO THIS DOCUMENT:
% -References will be automatically updated if any part is added, deleted, etc.
%  However, if a sub part is moved to a different part, its references must be
%  changed.
% -This document must be ratified by the House (as per the Constitution) if
%  changes are to be officialized.

\documentclass{article}
\providecommand{\RevisionInfo}{}
\usepackage{hyperref}
% Reformat section titles
\usepackage{titlesec}

% This package is useful for debugging label problems
% Comment out in final revision
%\usepackage{showkeys}

% Title page information
\title{Computer Science House Constitution}
\author{Computer Science House Constitution Committee}
% Last Modified Date
\newcommand{\datechanged}{Last Updated: \RevisionInfo}
\date{\datechanged}

% Fix margins
\setlength{\evensidemargin}{0in}
\setlength{\oddsidemargin}{0in}
\setlength{\textwidth}{6.5in}
\setlength{\topmargin}{0in}
\setlength{\textheight}{8.5in}

% Use \article for articles and \asection for sections of articles.
% Automatically provide labels with the same article or section title.
\newcommand{\article}[1]{\section{#1} \label{#1}}
\newcommand{\asection}[1]{\subsection{#1} \label{#1}}
\newcommand{\asubsection}[1]{\subsubsection{#1} \label{#1}}
\renewcommand{\thesection}{\Roman{section}}
\renewcommand{\thesubsection}{\arabic{section}.\Alph{subsection}}
\renewcommand{\thesubsubsection}{\arabic{section}.\Alph{subsection}.\arabic{subsubsection}}
\titleformat{\section}{\normalfont\Large\bfseries}{Article \thesection}{1em}{}
\titleformat{\subsection}{\normalfont\large\bfseries}{Section \thesubsection}{1em}{}

% Adding an \asubsubsection -- I feel dirty
%\setcounter{secnumdepth}{5}
\newcommand{\asubsubsection}[1]{\paragraph{#1} \label{#1}}
\renewcommand{\theparagraph}{\arabic{section}.\Alph{subsection}.\arabic{subsubsection}.\Alph{paragraph}}

% Adding \a(sub){3,4}section during merge of bylaws and articles -- I feel _really_ dirty
\setcounter{secnumdepth}{7}
\setcounter{tocdepth}{7}
\newcommand{\asubsubsubsection}[1]{\parindent=0em\subparagraph{#1} \label{#1}}
\renewcommand{\thesubparagraph}{\arabic{section}.\Alph{subsection}.\arabic{subsubsection}.\Alph{paragraph}.\arabic{subparagraph}}

\newcounter{asubsubsubsubsection}[subparagraph]
\renewcommand{\theasubsubsubsubsection}{\arabic{section}.\Alph{subsection}.\arabic{subsubsection}.\Alph{paragraph}.\arabic{subparagraph}.\Alph{asubsubsubsubsection}}
\newcommand{\asubsubsubsubsection}[1]{\parindent=0em\refstepcounter{asubsubsubsubsection}\par\textbf{\theasubsubsubsubsection\hspace{1em}#1 \label{#1}}}

% Headings
\pagestyle{myheadings}
\markright{{\rm CSH Constitution \hfill \datechanged \hfill Page }}

% Reference example:
%Test reference \ref{House Objectives} House Objectives.


\begin{document}
% Title
\maketitle

\tableofcontents

% ARTICLE I - INTRODUCTION
\article{Introduction}

\asection{Name}
The name of this Special Interest House shall be Computer Science House.

\asection{Derivation of Authority}
The Computer Science House shall recognize that it receives its right to function as a Special Interest House from the Center for Residence Life.

\asection{House Objectives}
The objectives of Computer Science House are:
\begin{enumerate}
	\item To enhance the education experience of its members
	\item To offer students educational programming with an emphasis in computers
	\item To provide a variety of services for its members, the RIT campus, and the Rochester Community
	\item To provide a friendly and comfortable living environment in the residence halls
\end{enumerate}

% ARTICLE II - CONSTITUTIONAL STRUCTURE
\article{Constitutional Structure}

\asection{House Charter}
The House Charter is a document drafted by the department of Residence Life, setting down the guidelines within which this House operates and from which it derives its authority.

\asection{Constitution}
The House Constitution is written and maintained by the House and defines the major aspects, goals, and governing structure of the house.
It is reviewed annually by the Department of Residence Life.

\asubsection{Non-Semantic Changes}
There are two methods for non-semantic change to the Constitution.
A Maintainer may approve any proposed change that does not affect the meaning of the document.
Alternatively, the change may be presented to the House for discussion followed by an Immediate One-Half Vote with fifty percent Quorum.

\asubsection{Semantic Changes}
Any semantic change to the Constitution requires the change to be proposed in writing for discussion at a House Meeting.
Any modifications made due to the discussion are added to the written proposal and the modified proposal is posted in the House during the week.
The final proposal is presented the following week and ballots are distributed for a Balloted Two-Thirds Vote with Two-Thirds Quorum \ref{Balloted Vote}.
The ballots are collected for a minimum of a forty-eight hour period.
The Constitution may be overridden by an Nine-Tenths Immediate Vote with eighty-five percent Quorum.

\asection{Maintainers}

\asubsection{Maintainer Qualifications}
Maintainers must be Active or Alumni in good standing.

\asubsection{Maintainer Expectations}
Maintainers are expected to:
\begin{itemize}
	\item Review changes to the Constitution for grammar, spelling, and internal consistency
	\item Keep a public record of changes to the Constitution
	\item Participate in discussion of proposals
	\item Facilitate the creation of new changes to the Constitution by other members
	\item Be knowledgeable about the Constitution
\end{itemize}
Failure to meet any of these expectations is grounds for revocation of Maintainer status by the Executive Board.

\asubsection{Maintainer Resignations}
Maintainers may resign at any time by notifying the current Executive Board and Maintainer group.

\asubsection{Maintainer Term}
Maintainer status lasts until it is resigned, or until it is revoked by an Executive Board Vote.
If a Maintainer no longer satisfies \ref{Maintainer Qualifications}, they lose Maintainer status.

\asubsection{Maintainer Selection}
Any member may nominate a qualified member for Maintainer status to the Executive Board for consideration.
The Executive Board may choose to approve or reject the nomination by an Executive Board Vote.

% ARTICLE III - GENERAL OPERATIONS OF THE HOUSE
\article{General Operations of the House}
\asection{Standard Operating Session}
The Standard Operating Session for Computer Science House is during the twenty-eight weeks of Fall and Spring semesters.
The Summer Sessions, Intersessions, Institute breaks, and all end of Semester Exam Weeks are considered non-standard operating sessions.
Unless explicitly stated otherwise, the requirements and expectations defined in the constitution are for the Standard Operating Session.

% ARTICLE IV - MEMBERSHIP
\article{Membership}
There are five major types of membership available to Computer Science House.
Each carries different qualifications, expectations, and privileges.
When describing the different memberships available, the following terms are used:
\begin{description}
	\item[Qualifications:] What qualifications an applicant needs to apply
	\item[Selection:] The process by which an applicant gains membership
	\item[Expectations:] The duties and responsibilities of House members
	\item[Privileges:] The benefits offered to House members
	\item[Evaluations:] The process by which a member is reviewed and assessed
	\item[Leave of Absence:] The process by which a member may leave House for a period of time
	\item[Resignations:] The process by which a member terminates House membership
	\item[Term:] The length of time the membership lasts
\end{description}

\asection{Introductory Membership}
\asubsection{Introductory Membership Qualifications}
Introductory Membership is open to all students at the Rochester Institute of Technology.
\asubsection{Introductory Membership Selection}
Applicants notify the House of their interest in membership by submitting an application to the Evaluations Director.
They must then undergo the selection process as defined in \ref{Selection Processes}.
\asubsection{Introductory Membership Expectations}
Introductory members are expected to meet all the requirements of the Introductory Process, as described in \ref{Expectations of an Introductory Member}.
\asubsection{Introductory Membership Privileges}
Introductory members receive the right to use Computer Science House facilities, to attend House functions, and to have housing priority over all persons who are not Computer Science House members.
\asubsection{Introductory Membership Evaluations}
Introductory members are evaluated on their performance during the introductory period.
The introductory evaluation process is described in \ref{Introductory Evaluation}.
\asubsection{Introductory Membership Leave of Absence}
After completion of the Introductory Packet, an Introductory member may request a leave of absence through the process described in \ref{Leave of Absence}.
\asubsection{Introductory Membership Resignations}
Introductory members may resign by submitting the request for termination of membership to the Chairperson, or Evaluations Director in writing before the completion of the introductory process.
\asubsection{Introductory Membership Term}
Introductory membership shall last until the end of the introductory process, at which time active membership is granted or membership is revoked.

\asection{Active Membership}
\asubsection{Active Membership Qualifications}
Active Membership is open to all students currently enrolled at the Rochester Institute of Technology who have passed the Introductory Evaluation and their most recent Membership Evaluation.
\asubsection{Active Membership Selection}
Qualifying members may self-select into Active Membership by paying dues to the Financial Director and notifying the Evaluations Director.
\\*\\*
Any Alumni Member in bad standing (\ref{Alumni Membership Selection}) may become an Active Member by notifying the Evaluations Director of their intent to participate.
The Evals Director will then bring them up for an Immediate Relative Majority vote with fifty percent Quorum at the next House meeting.
If the vote passes the Alumni is reinstated as an Active Member with Off-Floor status effective immediately.
\asubsection{Active Membership Expectations}
Active members are expected to be active participants in the House as defined in \ref{Expectations of House Members}.
\asubsection{Active Membership Privileges}
Active Members receive the right to:
\begin{itemize}
	\item Vote on all issues brought before the House
	\item Reside on the House
	\item Hold an Executive Board office on the House
	\item Use Computer Science House facilities
	\item Attend Computer Science House functions
	\item Receive priority for available housing on the House when returning from cooperative education
	\item Guaranteed housing in compliance with the Residence Life policies regarding Special Interest Houses and the Housing Selection Process (if they have On-Floor status)
\end{itemize}
Active Members currently on co-op forfeit their right to vote on house issues, and likewise do not count towards quorum.
Exceptions may be made at the discretion of the Evaluations Director.
\asubsection{Active Membership Evaluations}
Active Members are evaluated annually through the Evaluation Process as described in \ref{Evaluations Processes}.
\asubsection{Active Membership Leave of Absence}
An Active member may at any time request a leave of absence using the process described in \ref{Leave of Absence}.
For the duration of an absence, a member:
\begin{itemize}
	\item Forfeits their right to vote
	\item Does not count towards quorum
	\item Will not be required to attend House Meeting
\end{itemize}
\asubsection{Active Membership Resignations}
An Active member may resign by submitting, in writing, the reason for resignation to the Chairperson, or Evaluations Director.
Instead of forfeiting membership, Active members who resign may elect to become Alumni defined in \ref{Alumni Membership Selection}.
The resignation will take effect immediately and an announcement will be made at the following House Meeting.
\asubsection{Active Membership Term}
Active Membership shall last until the member: resigns or changes membership status.

\asection{Alumni Membership}
\asubsection{Alumni Membership Qualifications}
Alumni Membership is open to all former Active members of Computer Science House who passed at least one Membership Evaluation and departed for reasons other than revocation of House Membership.
\asubsection{Alumni Membership Selection}
Active members who depart house (i.e. resign) after passing the current operating session's Membership Evaluations are considered to be Alumni in good standing.
\\*\\*
Active members who depart house without passing the current operating session's Membership Evaluations are considered to be Alumni in bad standing.
This may be appealed to the Executive Board in order to pursue a different outcome.
\asubsection{Alumni Membership Expectations}
There are no expectations associated with the Alumni Membership status.
\asubsection{Alumni Membership Privileges}
Alumni Members shall receive the right to:
\begin{itemize}
	\item Use Computer Science House facilities
	\item Attend Computer Science House functions
\end{itemize}
\asubsection{Alumni Membership Evaluations}
Alumni Members are not subject to any evaluation.
\asubsection{Alumni Membership Resignations}
There are no resignations associated with Alumni Membership status.
\asubsection{Alumni Membership Term}
Alumni Membership shall last indefinitely or until the member chooses to pursue Active Membership.

\asection{Honorary Membership}
\asubsection{Honorary Membership Qualifications}
Honorary Membership is open to a person whom the House feels has contributed great personal effort to the House and is deserving of House recognition.
\asubsection{Honorary Membership Selection}
Any House member may nominate a candidate for Honorary Membership by submitting the name in writing to the Evaluations Director.
The Evaluation Director then begins the Honorary Membership Selection Process as defined in \ref{Selection Process for Honorary Members}
\asubsection{Honorary Membership Expectations}
Honorary Members are encouraged to remain in contact with the House.
\asubsection{Honorary Membership Privileges}
Honorary Members may advise in all issues brought before the House, use Computer Science House facilities, and attend Computer Science House functions.
\asubsection{Honorary Membership Evaluations}
Honorary Members are not subject to formal evaluations.
\asubsection{Honorary Membership Resignations}
An Honorary Member may resign by submitting in writing the reason for resignation to the Chairperson.
The resignation will be read at the following House Meeting and become effective at that time.
\asubsection{Honorary Membership Term}
Honorary Membership shall last until the member resigns.

\asection{Advisory Membership}
\asubsection{Advisory Membership Qualifications}
Advisory Membership is open to all members of the Rochester Institute of Technology professional, academic, or administrative staff.
\asubsection{Advisory Membership Selection}
The Executive Board may open nominations for Advisory Members.
The candidates then participate in the Advisory Selection Process as defined in \ref{Selection Process for Advisory Members}
\asubsection{Advisory Membership Expectations}
Advisors are encouraged to offer advice and assistance to the House in any capacity the Advisor is able to help.
Advisors are also encouraged to meet the House members and occasionally attend House activities.
\asubsection{Advisory Membership Privileges}
Advisory Members may advise in all issues brought before the House, use Computer Science House facilities, and attend Computer Science House functions.
\asubsection{Advisory Membership Evaluations}
Advisors are not subject to formal evaluations.
\asubsection{Advisory Membership Resignations}
An Advisor may resign by submitting in writing the reason for resignation to the Chairperson.
The resignation will be read at the following House Meeting and become effective at that time.
\asubsection{Advisory Membership Term}
Advisory Membership shall last until the member resigns.

\asection{Leave of Absence}
A leave of absence offers the option for members to take extended time away from their responsibilities to House for any number of personal issues including, but not limited to, physical illness, mental illness, or care giving for a sick family member.
Computer Science House recognizes the need for its members to take care of their personal well-being and will support them through the process of requesting a leave of absence.
House will also help to reacclimate any returning member back to the culture of House upon their return.
Upon approval of a leave of absence request, the Evaluations Director is notified and the member's leave is applied immediately.
A member is also allowed to be physically present and also be on a leave of absence.
\asubsection{Leave of Absence Request}
A leave of absence request must include the following information about the member: name, email, phone number, start date, expected return date, and a reason for the leave.
The request is then given to a staff member of Residence Life, excluding student employees, for approval.
A leave of absence start date can be backdated.
At no point does the Executive Board need to be informed of any details relating to the reason for the leave.
\asubsection{Leave of Absence Duration}
A leave of absence can be granted for up to the length of one semester.
\asubsection{Leave of Absence Extension}
A leave of absence can be extended by submitting a new request.
\asubsection{Leave of Absence Return}
A member may return from a leave of absence whenever they choose to.
When a member wishes to return, they must in writing notify a staff member of Residence Life, excluding student employees, to end their leave of absence.
\asubsection{Modified Evaluations}
A member may request their evaluation period to be extended by the duration of the leave or to end of the Standard Operating Session, whichever comes first.
\asubsection{Leave of Absence for an Executive Board Member}
The duties and responsibilities of an Executive Board Member on leave are assumed using the process defined in Section \ref{Appointment of an Interim Director} until the member returns from their leave.
Upon returning, the member may assume their position and the interim director is removed.
If at any point the member's leave is determined by the Executive Board to be detrimental to the operations of house, any Executive Board member may propose a Two-Thirds Immediate Vote amongst the Executive Board to require a resignation from the Member.
A two-thirds Quorum of the Voting Members of the Executive Board is required for the vote to be valid.

% ARTICLE V - EXECUTIVE BOARD
\article{Executive Board}
The Executive Board is the main governing body of the House.
Its purpose is to provide leadership and direction for the House, to oversee the day-to-day operations of the House, and to initiate and organize programs and projects for the House.
It is composed of the seven permanent directors and the Chairperson.
\\*\\*
There is one permanent directorship for each major aspect of the government of the House and each one is chaired by an Executive Board member.
Ad Hoc directorships are created on an as-needed basis.
They are generally very task oriented and are chaired by a House member.
A directorship has some jurisdiction in its area of interest and often is responsible for the day-to-day decisions regarding its area of interest.
Any large expenditures or large effect decisions must be brought before the entire House
\asection{Members of the Executive Board}
\asubsection{Voting Members}
\begin{itemize}
	\item Evaluations Director
	\item Social Director(s)
	\item Financial Director
	\item Research and Development Director(s)
	\item House Improvements Director
	\item Operational Communications Director
	\item House History Director
	\item Public Relations Director
\end{itemize}
\asubsection{Non-Voting Members}
\begin{itemize}
	\item Chairperson
	\item House Secretary
	\item Ad Hoc Director(s)
\end{itemize}
\asection{Closed Executive Board}
Closed Executive Board Meetings are open only to the Chairperson, Voting Members of the Executive Board, and those with the express permission of the aforementioned members.
A closed Executive Board meeting may be called at any time by any member of the Executive Board.
However, the Chairperson and at least two-thirds of the Voting Members of the Executive Board must be present for the meeting to be called.

\asection{Responsibilities}
\renewcommand{\theenumi}{\alph{enumi}} % For this section, we want items to use letters
\asubsection{Responsibilities of the Executive Board}
\begin{enumerate}
	\item To hold a weekly meeting specific to their responsibilities and submit notes to the House
	\item To meet, as an Executive Board, at least weekly during the Standard Operating Session, as defined in \ref{Standard Operating Session} to discuss and report the operations of the House
	\item To report pertinent information to House members at the following House Meeting
	\item To maintain records of the goals defined by each previous Executive Board
	\item To act as a Judicial Board as defined in \ref{Judicial}
	\item To review major projects, as defined in \ref{Expectations of House Members}
	\item To make the final vote regarding conditionals and appeals as defined in \ref{Membership Evaluation}
	\item To respect the privacy of House members confiding in the Executive Board, barring situations related to endangerment of oneself or others, sexual assault, or in the case of a Judicial Board
	\item To publish a document at the end of each semester to all Members stating House’s accomplishments of that semester
	\item To review and update the Constitution at the end of each Standard Operating Session, as defined in \ref{Standard Operating Session}.
		The constitution should remain up to date with current practices.
\end{enumerate}

\asubsection{Responsibilities of the Chairperson}
\begin{enumerate}
	\item To preside over Executive Board and House Meetings
	\item To exercise general supervision over the operations of the Executive Board
	\item To exercise general supervision over regular House activities
	\item To act as a liaison to the academic and administrative departments at RIT
	\item To act as a part of a Judicial Board as defined in \ref{Judicial}
	\item To cast tie-breaking vote in a split decision in an Executive Board Vote
\end{enumerate}

\asubsection{Responsibilities of the Evaluations Director}
\begin{enumerate}
	\item To preside over Evaluations Meetings
	\item To exercise general supervision over Evaluations operations
	\item To oversee the screening, interviewing, and acceptance or rejection of prospective House members
	\item To oversee the Semi-Annual Evaluations of current House members
	\item To collaborate with the Residence Life Advisor to determine room change selection and any changes of membership residency
	\item To act as a part of a Judicial Board as defined in \ref{Judicial}
	\item To prepare the House for Open Houses, tours, and special events
\end{enumerate}

\asubsection{Responsibilities of the Social Director}
\begin{enumerate}
	\item To preside over Social Meetings in which House members are encouraged to bring ideas for social events
	\item To exercise general supervision over Social operations
	\item To oversee the organization, initiation, and execution of House activities and events
	\item To ensure that there is a variety of activities for House members to participate in throughout the academic year
\end{enumerate}

\asubsection{Responsibilities of the Public Relations Director}
\begin{enumerate}
	\item To preside over Public Relations Directorship Meetings
	\item To operate House social media accounts intended to represent House as a whole
	\item To oversee the organization, initiation, and execution of House philanthropic events
	\item To preserve and improve the public image of House
	\item To collaborate with other Executive Board members to organize, initiate, execute, and advertise any events that are intended to be attended by persons whom have never been Active Members

\end{enumerate}

\asubsection{Responsibilities of the Financial Director}
\begin{enumerate}
	\item To preside over Financial Meetings in which House members are encouraged to bring ideas for fund-raising activities
	\item To supervise financial administrators and transactions involving house projects
	\item To maintain financial and inventory records of House capital and assets
	\item To plan and enforce a House budget
	\item To oversee House finances and generation of House funds
	\item To publish a semesterly House financial statement
	\item To supervise the collection of semesterly House dues
	\item To oversee the planning and execution of fundraising events
\end{enumerate}

\asubsection{Responsibilities of the Research and Development Director}
\begin{enumerate}
	\item To preside over Research and Development Meetings in which House members are encouraged to bring ideas for projects
	\item To exercise general supervision over Research and Development operations
	\item To oversee the planning, organization and construction of technical projects for the House's benefit
	\item To collaborate with the Operational Communications Director in an attempt to fulfill the House's need for technical equipment
	\item To provide seminars and tutorials to educate House members in technical areas
\end{enumerate}

\asubsection{Responsibilities of the House Improvements Director}
\begin{enumerate}
	\item To preside over House Improvements Meetings
	\item To exercise general supervision over the House Improvements operations
	\item To oversee the organization and construction of physical improvements to the House
	\item To oversee the general maintenance of the appearance of the House
\end{enumerate}

\asubsection{Responsibilities of the Operational Communications Director}
\begin{enumerate}
	\item To represent the Root Type Persons at Executive Board Meetings and at House Meetings
	\item To report the status of the House computer systems and network to the Executive Board and House members
\end{enumerate}

\asubsection{Responsibilities of the House History Director}
\begin{enumerate}
	\item To preside over House History Meetings
	\item To exercise general supervision over the House History operations
	\item To maintain, uphold, and promote house traditions
	\item To collaborate with the Social Director(s) to ensure that all the Active, Alumni, and Advisory Members are informed of upcoming events
	\item To oversee the production and distribution of a semi-annual newsletter
	\item To oversee the creation of a yearbook outlining House events for the year
\end{enumerate}

\asubsection{Responsibilities of the House Secretary}
\begin{enumerate}
	\item To ensure that minutes are recorded and posted for Executive Board Meetings
	\item To ensure that minutes are recorded and posted for House Meetings
	\item To oversee the maintenance of Executive Board records and documents
	\item To provide administrative assistance to the Executive Board
	\item To record all votes cast during House Meetings and Open Executive Board Meetings
\end{enumerate}

\asubsection{Responsibilities of Ad Hoc Directorships}
\begin{enumerate}
	\item Ad Hoc Directorships are responsible for the task for which they were created
	\item Any responsibilities budgets or finances for the Ad Hoc Directorship are placed upon the assigned director
\end{enumerate}

\asection{Operations}
\asubsection{Operations of the Financial Directorship}
\asubsubsection{Amount of House Dues}
The amount of dues for Active Members will be eighty dollars (\$80.00) per Academic Semester.
\asubsubsection{Collection of House Dues}
The collection period of house dues will be decided in conjunction with The Center for Residence Life and Housing Operations.
During the dues collection period, dues for on-floor members (for both semesters) are collected through the member’s RIT bill.
Dues for off-floor members are to be collected during this period as well by the Financial Director.
For any member who moves on or off-floor during the year, any difference between dues paid and dues owed should be collected.
Dues for any alumnus/alumnae in good standing are to be collected when intention to pay is expressed to the Financial Director.
\asubsubsubsection{Rules for Giving Exceptions and Exemptions for House Dues}
If a member is unable to pay dues upon request, they may appeal their situation to the Financial Director.
If the Financial Director deems their situation appropriate, they may grant specific extensions or reductions of the member's payment as they see fit.
If the Financial Director deems their situation inappropriate for extension, the member may appeal to the Executive Board.
If the Executive Board disagrees with the Financial Director, the Executive Board may reduce the required payment or extend the payment deadline.
If the Executive Board agrees with the Financial director, the member’s payment is considered delinquent and the member’s privileges are revoked until payment can be made.
In the case the Financial Director or Executive Board decides the situation is appropriate, the member's dues may be partially or fully excused, or the member may be given additional time to pay their dues. 
Additional time and reduction of dues may both be given to a member in an appropriate situation.
All dues must be collected in full before the annual membership evaluation (\ref{Membership Evaluation}).
After this date, dues collection is suspended until the start of the next academic year.
\asubsubsection{Breakdown of Dues for Directorship Budgets}
\begin{center}
\begin{tabular}[c]{|l c|}
\hline
Directorship Name & Percentage of Dues \\
\hline
\hline
Operational Communications & 20\% \\
\hline
Evaluations \& Selections & 5\% \\
\hline
House History & 10\% \\
\hline
House Improvements & 15\% \\
\hline
Research and Development & 20\% \\
\hline
Social & 20\% \\
\hline
Public Relations & 5\% \\
\hline
Accumulated & 5\% \\
\hline
\end{tabular}
\end{center}

The suggested total operational budget is \$8064.
This figure is based on yearly estimated on-floor member dues (\$160 x 72 members = \$11520) minus the 10\% reserved for Accumulated (Accum).
Money allocated for Accum and off-floor member dues is deposited into the CSH Account and saved.

\asubsubsection{Expenditure Approval}
All House expenditures must be approved by both the Financial Director and the appropriate Director (from whose budget the funds will be drawn).
Single expenditures may not exceed seventy-five dollars (\$75) and total expenditures may not exceed one-hundred dollars (\$100).
A total expenditure is defined as all the funds drawn for a specific event, project, piece of equipment, or service.
\\* \\*
If the above amounts are to be exceeded, the expenditure must be approved by an Immediate Relative Majority Vote at a House meeting before the funds may be appropriated.
At the Financial Director's discretion, a Spending Committee meeting may be held in order to approve any purchase exceeding one-hundred dollars (\$100), but not to exceed three-hundred dollars (\$300).
A Two-Thirds Immediate Vote with twenty-five percent Quorum is required to approve the purchase.
\\*\\*
For expenditures exceeding \$300 in total funding, a Complete Project Proposal must be presented to House at a House meeting.
This proposal includes the project budget, inventory of required resources, and a timeline for completion.
\\* \\*
If an appropriate Directorship cannot be determined for an expenditure, it is to be brought up for approval at House meeting as a Miscellaneous expenditure and approved by an Immediate Relative Majority vote with fifty percent Quorum, regardless of the amount.
This amount is to be directly subtracted from the general CSH account.
\\* \\*
Funds allocated for a project not spent by the end of the Standard Operating Session \ref{Standard Operating Session} in which they were approved are voided, unless otherwise specified during the original voting process.

% BY-LAW IV - OPERATIONS OF THE EVALUATIONS DIRECTORSHIP
\asubsection{Operations of the Evaluations Directorship}
\asubsubsection{The Introductory Process}
The Introductory Process is designed to provide an easy means for Introductory Members to meet existing House members, learn House history, demonstrate their involvement potential to House, and allow existing House members to evaluate them for Active Membership.
\asubsubsubsection{The Evaluation Period}
The Process will occur either once or twice per semester, and will last for six (6) weeks.
The Process shall be initiated by the Evaluations Director during either the first and second week of each semester.
The Executive Board may approve an extension to the Introductory Process or the Introductory Packet by an Executive Board Vote.
If an Introductory Process is extended, the Introductory Evaluation shall occur at the termination of the extended Process, which may be outside of the period defined in \ref{Introductory Evaluation}.
If deemed necessary by the Evaluations Director, or by an Executive Board Vote, the second intro process must begin within nine (9) days of the first process ending.
If a second Introductory Process is approved, the Evaluations Director may initiate a new Introductory Packet within the first or second week of the Introductory Process.
\asubsubsubsection{The Introductory Packet}
Each Introductory Process participant, after the first week, is given two (2) weeks to return a document that should contain a signature from all Active Members who have passed a Membership Evaluation, all Active Resident Members, all Introductory Resident Members, and ten (10) of any combination of the following:
\begin{itemize}
	\item Alumni members
	\item Honorary members
	\item Advisory members
\end{itemize}
At the discretion of the Evaluations Director, any Introductory Packet may be extended to accommodate extenuating circumstances.
\asubsubsubsection{Expectations of an Introductory Member}
Before the end of the Process, an Introductory Member is expected to:
\begin{itemize}
\item Attend all House meetings during the process
\item Complete the Introductory Packet
\item Attend at least one House directorship meeting for each week of the process (including permanent and Ad Hoc directorship meetings)
\item Attend at least one House social event during the process
\item Attend at least two seminars relating to computing or electronics
\item Sign a copy of the Membership Agreement describing a House Member’s responsibilities and expectations
\end{itemize}

\asubsubsubsection{Introductory Evaluation}
The Introductory Evaluation is the process by which House chooses which Introductory Members to extend an offer of full Resident or Non-Resident Membership.
Occurs at the conclusion of the final week of the Process.

\asubsubsubsubsection{Voting}
On a member by member basis, this evaluation process determines if each Introductory member has successfully completed the Introductory Process requirements \ref{Expectations of an Introductory Member}.
An Immediate Relative Majority Vote with two-thirds Quorum is required for the evaluation to be valid.
For an Active Member to be eligible to vote during the evaluation process they must meet the requirements outlined in \ref{Expectations of Voting Members}.
Neither absentee nor proxy votes are allowed.
House may choose any of the Outcomes for each member.
\asubsubsubsubsection{Expectations of Voting Members}
The following requirements must be met by the beginning of the Introductory Evaluation for an Active Member to be allowed to vote. 
Any of these requirements may be waived by the Evaluations Director or an Executive Board Vote.
\begin{itemize}
\item Attend all House meetings during the process
\item Attend at least one House directorship meeting for each week of the process (including permanent and Ad Hoc directorship meetings)
\item Attend at least one House social event during the process
\item Attend at least two seminars relating to computing or electronics
\end{itemize}

\asubsubsubsubsection{Outcomes}
\renewcommand{\theenumi}{\alph{enumi}} % For this section, we want items to use letters
\begin{enumerate}
	\item Introductory Members may be offered Active Membership (\ref{Active Membership}) provided they meet the requirements described in \ref{Expectations of an Introductory Member}, at the discretion of House.
	\item If an Introductory Member fails to meet the requirements, their membership will be revoked.
		In addition, the participant is asked to find alternative housing as soon as possible in accordance with all applicable Residence Life policies regarding room changes.
	\item An Introductory member may be given a conditional to complete as a means of making up for missing requirements.
		A conditional may be proposed by any member present at the Introductory Evaluation and, if it is approved by the Evaluations Director, is then voted on by House.
		Each conditional consists of a set of additional requirements and a deadline for completing them.
		When the deadline expires, the conditional will be brought before the Executive Board who will assess its completeness.
		A conditional member who does not meet these additional requirements forfeits membership.
\end{enumerate}
\asubsubsection{Selection Processes}
\renewcommand{\theenumi}{\alph{enumi}} % For this section, we want items to use letters
\asubsubsubsection{Selection Process for Students Attending RIT for at Least One Semester}
\begin{enumerate}
	\item The applicant submits an application to the Evaluations Director for review.
	\item The applicant participates in an informal interview with exactly three current Active, Alumni, or Honorary Members.
	\item The application materials are then reviewed at an Evaluations meeting.
	Then an Immediate Relative Majority Vote of attending members is held on whether or not to accept the person as an Introductory Member.
\end{enumerate}
Note: Most membership privileges do not initiate until successful completion of the introductory process.
This means that until the member has passed the Introductory Evaluation, the member does NOT have the right to vote on House issues and does NOT count towards quorum.
The member does, however, have the right to use Computer Science House's facilities.
The member is not required to pay dues during the Introductory Process.
\\* \\*
Additional Note: No hazing shall occur at any time during the Selection Process in accordance with the New York State Hazing Laws.
\asubsubsubsection{Selection Process for First Semester and Entering RIT Students}
\begin{enumerate}
	\item During Spring semester, the Evaluations Director selects a group of Active Members to form a Selections Committee.
		The Selections Committee reviews applications and conducts interviews in accordance with the process prescribed by Residence Life.
	\item A subset of applications will be selected to move onto the House Fall Semester, are granted full membership privileges (including On-Floor status), and must participate in the Introductory Process as defined in \ref{The Introductory Process}.
		An additional subset will be given Off-Floor status, but otherwise receive the same privileges and responsibilities.
\end{enumerate}
Note: Most membership privileges do not initiate until successful completion of the introductory process.
This means that until the member has passed the Introductory Evaluation, the member does NOT have the right to vote on House issues and does NOT count towards quorum.
The member does, however, have the right to use Computer Science House's facilities.
The member is not required to pay dues during the Introductory Process.
\\* \\*
Additional Note: No hazing shall occur at any time during the Selection Process in accordance with the New York State Hazing Laws.
\asubsubsubsection{Selection Process for Honorary Members}
\begin{enumerate}
	\item A House Member submits to the Evaluations Director a nomination for a person they feel is deserving of Honorary Membership.
	\item The Evaluations Director performs some preliminary research on the candidate and presents the findings.
	\item The Executive Board decides whether or not to present the nomination to the House for a secret ballot House vote.
		If the Executive Board decides not to present the nomination to the House, the Selection Process ends and the candidate does not become an Honorary Member.
	\item The nomination is presented at a House Meeting for discussion and a Two-Thirds Balloted Vote with two-thirds Quorum.
		Ballots are distributed and voting must remain open for a minimum of forty-eight hour period.
	\item The candidate is notified of their selection as an Honorary Member and presented with the honor.
\end{enumerate}
\asubsubsubsection{Selection Process for Advisory Members}
\begin{enumerate}
	\item When there is a perceived need, the Executive Board may open nominations for Advisory Members.
		After an announcement at House Meeting, during a 72-hour period, any House Member may submit a nomination.
	\item After the close of the nomination collection period, the Executive Board will arrange some means for the House to meet with the nominees.
	\item A discussion of the candidates will be held at the following House meeting.
	\item Ballots are distributed for each nominee as a One-Half Balloted Vote with fifty percent Quorum.
		Voting must remain open for a minimum of a forty-eight hour period.
	\item All candidates selected are notified of their acceptance as House Advisors and asked to accept or decline the selection.
\end{enumerate}

\asubsubsection{Evaluations Processes}
During the academic year, a House member is evaluated by the Membership Evaluation.
The Membership Evaluation is responsible for determining those individuals who will have the option to continue Active Membership in the following year.
Semi-Annual Evaluations Meetings are open only to current Active Members.
All Executive Board members are expected to attend the Semi-Annual Evaluations Meetings to assist in the evaluations of House members.
\\* \\*
At the beginning of any Evaluation Process listed herein, the Evaluations Director must read the sections of the Computer Science House Constitution used during the respective Evaluation Process.
It is incumbent upon each House member to provide the Evaluation Director with whatever information they feel is necessary to ensure an accurate evaluation.
\asubsubsubsection{Membership Evaluation}
The Membership Evaluation process occurs once per academic year.
It is performed as part of the Evaluation Process that takes place during the spring semester to comply with RIT Housing deadlines.
\asubsubsubsubsection{Voting}
Any Active Member who has completed all of the requirements as defined in \ref{Expectations of House Members} at the beginning of the Membership Evaluation passes their Membership Evaluation without being voted on or evaluated by the quorum.
All Active Members who have not recieved an exemption from the Executive Board prior to the Membership Evaluation will be evaluated on a Member by Member basis by a quorum of Active Members.
The Membership Evaluation shall hold members to the objective requirements defined in \ref{Expectations of House Members} and determine which members may continue as an Active Member in the following Standard Operating Session.
Exceptions to the requirements may be made, as House may choose any of the Outcomes for each Member, even if the Member being evaluated has not completed all of their requirements.
These requirements must be completed before the Membership Evaluation occurs.
A Secret Immediate Relative Majority vote with two-thirds Quorum is required for the evaluation to be valid.
Neither absentee nor proxy votes are allowed.


\asubsubsubsubsection{Outcomes}
Members who pass Membership Evaluation have the option to participate as an Active Member in the following year and remain on the upcoming roster if applicable.
\\* \\*
If a member fails and has never passed a Membership Evaluation in the past, their membership will be revoked immediately.
If the member has previously passed a Membership Evaluation they will move to Alumni Membership status at the end of the Standard Operating Session (\ref{Alumni Membership Selection}).
In either case, the member forfeits their ability to be included on the following year’s roster.
\\* \\*
A member may be given a conditional to complete as a means of making up for missing requirements.
A conditional may be proposed by any member present at the Evaluation.
If the conditional is approved by the Evaluations Director, it is then voted on by House as described in \ref{Voting}.
Each conditional consists of a set of additional requirements a deadline for completing them.
When the deadline expires, the conditional will be brought before the Executive Board who will assess its completeness.
A conditional member who does not meet these additional requirements will have failed the evaluation.
\asubsubsubsection{Appeals Process}
If a member disagrees with the outcome of any evaluation, (e.g. is not asked to return to the House for the following year) and wishes to appeal the decision, they may do so as stated in \ref{Appeals}.
\\* \\*
If the member is still unsatisfied after being heard by the Executive Board, the appeal may be brought to the attention of the ResLife Advisor.

\asubsubsection{Expectations of House Members}
Each member is required to pay House dues as stated in \ref{Collection of House Dues}, attend all House Meetings and Executive Board candidate speeches, and attend at least fifteen (15) of the House directorship meetings (including permanent and Ad Hoc directorship meetings) for each Academic Semester in which they are an Active Member.
\\* \\*
The member must participate significantly in at least one major project during the academic year.
The member is required to submit a description for this major project to the Executive Board for approval.
As an option to this requirement, the member may instead assist on a large number of House activities and projects.
However, it is to be understood in advance by the member that this option requires a great deal of participation throughout the year.
This participation will be evaluated by the Executive Board.

\asubsubsection{Housing Status}
All Active and Introductory Members have a housing status.
This status indicates their right to priority housing on the floor.
Any Alumni Member in good standing will maintain their previous On Floor status upon a return to Active membership.
\asubsubsubsection{On-Floor Status}
Members with On-Floor status are eligible to live on the floor.
To be granted On-Floor status, members may notify the Evaluations Director that they would like to come up for a vote.
The Evaluations Director will then bring them up for a One-Half Immediate vote among attending Active members at the next evaluations meeting.
If the Evaluations Director determines that too few members are present to hold a vote, the member will be brought up for a vote at the next evaluations meeting.
\asubsubsubsection{Off-Floor Status}
Members with Off-Floor status do not have the right to live on the floor.
They still have access to all other privileges associated with their membership, and may still accumulate Housing Priority Points.

\asubsubsection{Housing Priority System}
The Housing Priority System is a means for determining the priority a House member has in a Housing issue such as Single Room Assignment or the Assignment of Available Housing.
The member with top priority is the member with on-floor status and the most Housing Priority Points.
Housing Priority Points are accumulated once per Operating Session at the conclusion of the Membership Evaluation. Each Active Member who passes the Membership Evaluation is granted two (2) Housing Priority Points.
\\* \\*
In the event of a tie, the members will be approached simultaneously and if they are unable to decide fairly between themselves, the assignment of priority will be made by random selection of the tied members.
\asubsubsubsection{Single Room Assignments}
When a single room on the House becomes available it is offered to the member who carries the highest Housing Priority as defined in \ref{Housing Priority System}.
In the event of a tie, the members will be approached simultaneously and if they are unable to decide fairly between themselves, the assignment of priority will be made by random selection of the tied members.
If that House member declines the option, it will be offered to the member with the next highest Housing Priority.
This process continues until a member selects to move into the single room.
Once in a single room, a House member retains the assignment until voluntarily giving it up.
It should be noted that members selecting this option must agree to any additional charges applied by the Department of Residence Life for residing in a single room.
\asubsubsubsection{Double Rooms as Single rooms}
During the third week of each semester, if there is no waiting list for residency on the House, any vacant rooms will be offered to House members as single rooms assignments according to the method as described in \ref{Single Room Assignments}.
It should be noted that this does not mean members will be relocated into empty spaces so that the member with top priority is offered the single.
This only applies if there is a totally vacant room.

\asubsection{Operations of the Operational Communications Directorship}
The Operational Communications Directorship is responsible for overseeing the implementation of maintenance and upgrades to the House Computer Systems Network.
It is a self-governing committee making decisions by Immediate Relative Majority vote (two-thirds Quorum) of Current Root Type Persons.
Membership is composed of all Root Type Persons.

\asubsubsection{Selection of a Root Type Person}
\renewcommand{\theenumi}{\alph{enumi}} % For this section, we want items to use letters
\begin{enumerate}
	\item Nominations are taken from the Root Type Persons and Prior Root Type Persons meeting the selection criteria in \ref{Qualifications of a Root Type Person}.
	\item Each candidate is given a minimum of twenty-four hour period to accept or decline the nomination.
	\item A list of all nominees who have accepted is presented to the Executive Board for approval.
		This Executive Board Meeting is closed to the Executive Board Members, Root Type Persons, and House members with explicit invitation from the Executive Board.
	\item If an Executive Board member is a candidate for the office in discussion, the member is absent and their vote is abstained.
		An Executive Board Vote is taken to determine whether the nominations of the Root Type Persons are accepted.
\end{enumerate}

\asubsubsection{Qualifications of a Root Type Person}
\renewcommand{\theenumi}{\alph{enumi}} % For this section, we want items to use letters
\begin{enumerate}
	\item Candidates must have been an On-Floor Active Member in good standing within the last twelve months, or been an Off-Floor Active Member and have lived on the floor within the last twelve months.
	\item Candidates may be granted an exemption by current Root Type Persons.
		Such an exemption may be withdrawn at any time by the current Root Type Persons.
	\item Prior Root Type Persons are those members who are no longer current Active Members and have not been granted an extension by the current Root Type Persons.
		Prior Root Type Persons are not guaranteed access to the current root passwords and other authentication tokens.
	\item The current Root Type Persons may from time to time draft rules and regulations specifying the rights and privileges of Prior Root Type Persons.
\end{enumerate}

\asubsubsection{Creation of Accounts}
Root Type Persons have the authority to manage user accounts for House systems.
Before introductory, active, or alumni members may receive an account they must:
\renewcommand{\theenumi}{\alph{enumi}} % For this section, we want items to use letters
\begin{enumerate}
	\item Sign the Code of Conduct sheets pertaining to the responsible utilization of Computer Science House and Rochester Institute of Technology facilities.
	\item Obtain greater than or equal to 60\% (rounded up to the nearest whole person) of required signatures (excluding those of Resident members who have not passed a Membership Evaluation) in the Introductory Packet or successfully complete Introductory Evaluations.
\end{enumerate}
Accounts for honorary and advisory members may be created at the discretion of the Root Type Persons.

\asubsubsection{Code of Conduct}
The Code of Conduct located at \url{https://github.com/ComputerScienceHouse/CodeOfConduct} is the canonical Code of Conduct for Computer Science House accounts.
\\* \\*
Members are bound to the Code of Conduct revision that they sign when initially creating their account.
Members may sign a more recent revision of the Code of Conduct to update their agreement.
\asubsubsubsection{Changes}
The following process is used for making changes to the Code of Conduct document.
\begin{enumerate}
	\item A Root Type Person drafts a change to the Code of Conduct and makes publicly available both the summary and difference file of the change.
	\item The change is proposed at a House Meeting and is discussed at the same House Meeting.
	\item The final proposal is presented at the next House Meeting and ballots are distributed for a One-Half Balloted House vote (fifty percent Quorum) as described in \ref{Balloted Vote}.
\end{enumerate}


\asection{Qualifications}
\asubsection{Qualifications to be the Chairperson, Evaluations Director}
\begin{enumerate}
	\item Candidates must be Active Members during the term of office
	\item Candidates must reside on floor during the term of office
	\item Candidates must have at least one full semester of House membership as an Active Member
	\item Elected or selected candidates may not hold a simultaneous voting Executive Board position, and must therefore resign their current position, or the position of Chairperson, should they be elected
\end{enumerate}
\asubsection{Qualifications to be the Social Director(s), Financial Director, Research and Development Director(s), House Improvements Director, House History Director, Public Relations Director}
\begin{enumerate}
	\item Candidates must be Active Members during the term of office
	\item Candidates must have at least one full semester of House membership as an Active Member
	\item Elected or selected candidates may not hold two simultaneous voting Executive Board positions, and must therefore resign their current position or decline a second position should they be elected or selected to a second voting position
\end{enumerate}
\asubsection{Qualifications to be the Operational Communications Director}
\begin{enumerate}
	\item Candidates must be a current Root Type Person \ref{Operations of the Operational Communications Directorship}.
	\item Candidates must be a current Active Member.
	\item A Root Type Person cannot be the director if they currently hold any other voting Executive Board Position.
\end{enumerate}

\asubsection{Qualifications to be the House Secretary or an Ad Hoc Director}
\begin{enumerate}
\item Candidates must be Active Members
\end{enumerate}

\asubsection{Waiving of Qualifications}
\begin{enumerate}
  \item House may choose to waive the following subset of the Qualifications for Executive Board positions (defined under \ref{Qualifications}) for all Candidates:
    \begin{enumerate}
      \item Candidates must reside on floor during the term of office.
      \item Candidates must have at least one full semester of House membership as an Active Member.
      \item Social and Research and Development are the only voting Executive Board positions that allow for Dual Directorship.
    \end{enumerate}
  \item When a waiver is proposed, the Chair of the Vote shall state which Executive Board position and Qualification is being voted upon, and any nominated Candidates not qualified under \ref{Qualifications} who would become qualified if the waiver passes.
  \item A Three-Quarters Immediate Vote with three-quarters Quorum is then taken to determine whether the proposed waiver shall take effect.
  \item Each vote to waive a Qualification must only apply to a single Qualification for a single Executive Board position.
  \item Waivers always apply to all Candidates for the Executive Board position.
\end{enumerate}

\asection{Selection}
\asubsection{Dual Directorship}
Social and Research and Development are the only voting Executive Board positions that allow for Dual Directorship.
If two candidates elect to run as a Dual Directorship, their names are placed together on a single line of the election ballot.
If they are also nominated as a normal directorship or as a dual directorship with another House member, their votes are not cumulative.
Each different nomination must be a separate entry on the election ballot.
Non-voting Executive Board positions (i.e. Ad Hoc Directors) are not restricted to single or dual directorship.

The following special cases cover the operation of a dual directorship:
\begin{itemize}
	\item If one of the members in a dual directorship resigns the directorship, or for any other reason ends the term of office, the other member in the dual directorship must also step down and the office becomes vacated.
		The vacated office is then handled like any other vacated office; see \ref{Executive Board Resignations}.
	\item During an official Executive Board Vote, each dual directorship member's vote counts for one-half vote in the tallying of votes.
		The members of the dual directorship need not vote the same way in a vote.
	\item A member of a dual directorship may not hold any other Executive Board position.
	\item When attendance requirements call for the dual directorship position to be present, both directorship members must be present to fulfill the requirements.
	\item Dual directorships must have both members present for to count towards Quorum in an Executive Board Vote.
\end{itemize}
\asubsection{Selection of the Chairperson, Evaluations Director, Social Director(s), Financial Director, House Improvements Director, House History Director, Research and Development Director(s), Public Relations Director}
\begin{enumerate}
	\item The opening of the Executive Board position(s) is announced at a House meeting and nominations for the position are taken for a minimum of a seventy-two hour period.
		Any House member may nominate any member or pair of members where appropriate for a Directorship.
	\item The candidates will be notified of their nomination.
		Each candidate is given a minimum of twenty-four hour period to accept or decline the nomination.
		A list of all nominees who have accepted their nominations will be posted shortly thereafter.
	\item The date and time of candidate speeches will be announced by a member of the Executive Board at least five (5) days prior to the event. 
		All Active Members are expected to attend candidate speeches in accordance with \ref{Expectations of House Members}.
		If a member cannot attend due to a scheduling conflict, they may provide a reason for their absence to the Evaluations Director, who will record the reason with the absence.
	\item Each candidate will be given an equal amount of time before the House to present their platform of candidacy.
	\item Ballots will then be distributed for a Balloted Vote \ref{Balloted Vote} and voting will be open for a minimum of a forty-eight hour period.
		The Ballots will list, in random order, all of the candidates who are qualified for a given office with a means to indicate the selection of one of the candidates.
		In addition, an area will be provided to indicate a write-in selection of a candidate.
	\item At the end of the voting period, the Chairperson will terminate voting.
	\item The winners are determined via the process described in \ref{Ranked Choice}.
		A fifty-percent Quorum is required for the election to be official.
		All winners are notified of their election.
		If the position is currently vacant, the winner immediately assumes office.
		If not, the winner will assume office at the end of the current term (\ref{Term}).
		Any office whose winner declines the election, or whose winner does not fulfill the requirements of the elected office, shall have their votes redistributed in the same process as a loser in \ref{Ranked Choice}.
\end{enumerate}
\asubsection{Selection of the Operational Communications Director}
\begin{enumerate}
	\item A candidate for the Operational Communications Directorship shall be chosen by the current Root Type Persons.
	\item The candidate is given a minimum of twenty-four hours to accept or decline the nomination.
	\item The candidate is presented to the Executive Board for approval.
		This Executive Board meeting is closed to the Executive Board Members, Root Type Persons, and House members with explicit invitation from the Executive Board.
	\item All current Executive Board Voting Members must be present during the discussion and voting period unless that member is a candidate for the position, in which case the member is absent and their vote is abstained.
		An Executive Board Vote is taken to determine whether the candidate is selected for the position.
\end{enumerate}
\asubsection{Selection of Ad Hoc Directors}
\begin{enumerate}
	\item When the Executive Board or a group of House members feels an Ad Hoc Directorship is necessary, they present their plans to the Executive Board.
  An Executive Board Vote is taken to determine whether directorship status is granted.
	\item When the Ad Hoc Directorship is granted status, a director is appointed, and duties, budgetary, and membership considerations are defined.
\end{enumerate}
\asubsection{Selection of the House Secretary}
The Executive Board may choose to select a current voting Executive Board member, or to select any interested Active member, as House Secretary.
The selection process can be an informal appointment, or follow an election process similar to other voting Executive Board positions.

\asection{Executive Board Resignations}
An Executive Board Member may resign their position by submitting in writing the reason for resignation to the House Chairperson.
The resignation will be read at the following House Meeting and become effective at that time.
The office will become vacant and the selection process for a new member as described in section \ref{Appointment of an Interim Director} will begin at that time.
Following a resignation, the Selection process, as defined in \ref{Selection}, of a replacement for the position vacated must begin within two weeks from when the resignation took place.
To postpone such a Selection, the Chairperson may chair an Immediate Relative Majority Vote \ref{Immediate Vote} \ref{Relative Majority} during a House Meeting prior to the beginning of the Selection process to delay the Selection process by a specified amount of time.

\asection{Appointment of an Interim Director}
The duties and responsibilities of a vacated office are assumed by the Chairperson or by an Active member that is appointed at the Chairperson's discretion until the new member takes office.
The vote of a vacated Executive Board position shall be cast as an abstention in all Executive Board matters where this vote is required to be cast.

\asection{Impeachment}
\begin{enumerate}
	\item Impeachment of any Executive Board Member may be initiated by petition, in writing, consisting of a minimum number of signatures of current House members equaling or exceeding one-third the Number of Possible Votes as defined in \ref{Total Number of Possible Votes}.
	\item The impeachment petition is then presented at an Executive Board Meeting.
		The member(s) initiating the petition present their case to the Executive Board.
		The Executive Board then questions the accused member of the allegations.
	\item An Executive Board Vote is taken on the impeachment petition, with all voting members present except the accused member who must be absent and whose vote counts as an abstention, to determine if the allegations stated in the petition are legitimate grounds for impeachment.
		If the majority of Executive Board votes are negative, the petition and impeachment proceedings are dismissed.
		This vote may be overridden by an impeachment petition consisting of the grounds for impeachment and a minimum of number of signatures of current House members equaling or exceeding two-thirds the number of all House members currently eligible to vote.
	\item If the majority of the Executive Board votes are positive, or the negative vote was overridden, the petition is presented at the following House Meeting and the accused and accuser(s) again present their cases.
	\item Ballots will then be distributed and a secret ballot House vote shall be held for a minimum of a forty-eight hour period to determine whether or not the member should be removed from office.
		Votes shall be collected and counted for a Two-Thirds Balloted Vote described in \ref{Balloted Vote}.
	\item A two-thirds Quorum is necessary for the vote to be official.
		If the resolution passes, the accused officer is relinquished of their position and any benefits thereof and this position is treated like any other vacated position.
		If a quorum cannot be reached after two attempts, or the percentage of affirmative votes does not equal or exceed the minimum, impeachment proceedings are dismissed.
		A new selection and interim duty fulfillment procedure is followed similar to that of a resignation; see \ref{Executive Board Resignations}.
	\item House Secretary can be impeached according to the above process, or may be dismissed by an Executive Board Vote.
\end{enumerate}

\asection{Term}
\begin{enumerate}
\item The Election Process for the following year's Executive Board members shall begin in the middle of Spring semester.
\item The newly selected officers shall begin their terms on June 1 of that year and their terms shall end on May 31 of the following year.
\item The term of an officer will be abbreviated due to resignation, impeachment, or change in membership status.
\item Officers elected or selected during the course of the year due to an abbreviated term of the previous officer shall hold office until the end of the normal term.
\item When the task of an Ad Hoc Directorship has been completed, the directorship dissolves.
	When an Ad Hoc Director resigns, the directorship dissolves and must be reinstated with a new director.
\end{enumerate}

\asection{Appeals}
Decisions made by the Executive Board may be appealed and overturned.
To initiate an appeal, a member must have the support of three voting members of the Executive Board, or a petition with the signatures of one-third of Active Members.
After the appeal is presented, an Executive Board Vote is taken to determine whether or not to overturn and reevaluate the decision.
If the vote passes, the Executive Board may discuss and make a new ruling by another Executive Board Vote.

% ARTICLE VI - VOTING
\article{Voting}
This section outlines the different types of votes and ballots used to make House decisions and defines relevant terminology.
\asection{Definitions}
\asubsection{Eligible Votes}
The number of Active members eligible to vote on the issue.
\asubsection{Votes Cast}
The number of ballots cast in a vote, excluding Abstentions.
\asubsection{Quorum}
The minimum Votes Cast for vote to be considered valid.
Any member present for an Immediate Vote or given a ballot who does not explicitly cast their vote is counted as an Abstention.
A Quorum is expressed as a fraction or percentage of Eligible Votes (rounded up to the nearest vote) unless otherwise specified in the text of the vote.
\asubsection{Proxy Vote}
A vote cast by one member on behalf of another.
Proxy Votes are only permissible at the discretion of the Chair of the Vote, and may be disallowed in the text describing the vote.
Proxy Votes must be documented in writing and signed by the absent member.
The count of all Proxy Vote must be recorded and announced in all votes.
\asubsection{Abstention}
An Abstention is a neutral vote that counts towards the Quorum but not towards determining the outcome of a vote.
A means to abstain must always be provided in a vote.
\asubsection{Vote Counters}
The Chair of the Vote is a vote counter and will additionally select two other members to count votes.

\asection{Voting Procedures}
\asubsection{Balloted Vote}
\asubsubsection{Method of Vote}
Votes are cast on ballots that provide a means to indicate every possible option in the vote.
\asubsubsection{Voting Period}
For constitutional modification, candidate selection, and officer removal votes, the voting period must be at least forty-eight (48) hours in length.
For any other type of vote, the voting period must be at least twenty-four (24) hours.
The minimum length of the voting period may be explicitly lengthened, but never shortened, in the text describing the actual vote.

The voting period opens once ballots are distributed to each House member eligible to cast a vote.
At the end of the voting period, the Chair of the Vote collects the ballots, closing the voting period.
The Vote Counters then tally the results.

\asubsection{Immediate Vote}
\asubsubsection{Method of Vote}
The Chair of the Vote will state all possible ways to vote, then call out each possibility one at a time.
The chairing member will count the number of members casting their immediate vote for that possibility.
Any members that count towards Quorum that do not explicitly cast a vote will have their vote counted as an Abstention.
\asubsubsection{Voting Period}
An Immediate Vote lasts as long as it takes for all votes to be tallied.
\asubsubsection{Secret Immediate Vote}
An Immediate Vote in which votes are kept anonymous.
\asubsubsection{Executive Board Vote}
In an Executive Board Vote, only Voting Members of the Executive Board are eligible to vote.
Unless otherwise specified, Executive Board votes are Immediate Relative Majority votes with a fifty percent Quorum.

\asubsection{Batch Vote}
When a Batch Vote is called for by the Chair of the Vote, a subset of the voting docket may be amended to a single vote. % precedent: a non-empty subset, nerrrrrds.
A Two-Thirds Immediate Vote is required to allow a Batch Vote to take place.
If the call for Batch Vote passes, the subset may then be voted on, otherwise the docket remains unchanged.

\asection{Approval Criteria}
Approval Criteria are the rules that determine whether an option in a vote has passed.
\asubsection{Relative Majority}
In a Relative Majority vote, an option in the vote passes if the number of votes cast for that option is larger than the number of votes cast for every other option individually.
\asubsection{Fractional}
In a Fractional vote, an option in the vote passes if the number of votes cast for that option exceeds a specified fraction of the Votes Cast, rounded up to the nearest vote.
The fraction is specified in the text describing the vote (eg, "Two-Thirds Vote").
\asubsection{Ranked Choice}
In a Ranked Choice vote, voters rank the options by placing a '1' by their first choice, a '2' by their second choice, and so on, until they no longer wish to express any further preferences or run out of options.
The winning option is selected outright if it gains more than half the votes cast as a first preference.
If not, the option with the fewest number of first preference votes is eliminated and their votes move to the second preference marked on the ballots.
This process continues until one option has half of the votes cast and is elected.

\asection{Ties Between Vote Options}
\asubsection{With Pass/Fail}
If the number of votes cast for the pass option equals the number of votes cast for the fail option, then the vote has failed.
\asubsection{With Multiple Options}
If only one option may pass, then the vote must be recast or tabled at the discretion of the Chair of the Vote.
In the event of a tie in an Executive Board Vote, the Chairperson may cast the tie-breaking vote.

% ARTICLE VII - JUDICIAL
\article{Judicial}
The Executive Board may approve a request (from a member) for a Judicial proceeding when an official clarification of the Constitution is required, there is a conflict among House Members, or a House interest needs resolution.
\asection{Formation of a Judicial Board}
A Judicial Board is made up of the Chairperson, the Evaluations director, and an additional voting member of the Executive Board.
If either the Chairperson or Evaluations Director are deemed biased or unfit for a position on the Judicial Board, they will be replaced by another voting member of the Executive Board by means of an Executive Board Vote of the remaining members.
\asection{Judicial Investigation}
The Judicial Board will be responsible for making all necessary inquiries into the matter of the request brought to the Judicial Board.
\asection{Judicial Ruling}
The Judicial Board may present an official ruling following the Judicial Investigation.
This ruling may be appealed in the same manner as an Executive Board decision \ref{Appeals}.

% ARTICLE VIII - ADHERENCE TO UNIVERSITY POLICIES
\article{ADHERENCE TO UNIVERSITY POLICIES}
\asection{Anti-Hazing}
\asubsection{RIT Hazing Policy (RIT Student Conduct Process; IV. RIT Code of Conduct; 14. Hazing/Failure to Report
Hazing)}
Hazing/Failure to Report Hazing.
Behavior, regardless of intent, which endangers the emotional, or physical health and safety of a Student for the purpose of membership, affiliation with, or maintaining membership in, a group or Student Organization. Hazing includes any level of participation, such as being in the presence, having awareness of hazing, or failing to report hazing.
Examples of hazing include, but are not limited to, beating or branding, sleep deprivation or causing excessive fatigue, threats of harm, forcing or coercing consumption of food, water, alcohol or other drugs, or other substances, verbal abuse, embarrassing, humiliating, or degrading acts, or activities that induce, cause or require the Student to perform a duty or task which is not consistent with fraternal law, ritual or policy or involves a violation of local, state or federal laws, or the RIT Code of Conduct.
\asubsection{NY State Hazing Law}
\asubsubsection{§ 120.16 Hazing in the first degree.}
A person is guilty of hazing in the first degree when, in the course of
another person's initiation into or affiliation with any organization, he intentionally or recklessly engages in
conduct which creates a substantial risk of physical injury to such other person or a third person and thereby
causes such injury. Hazing in the first degree is a class A misdemeanor.
\asubsubsection{§ 120.17 Hazing in the second degree.}
A person is guilty of hazing in the second degree when, in the course of another person's initiation or affiliation with any organization, he intentionally or recklessly engages in conduct which creates a substantial risk of physical injury to such other person or a third person.
Hazing in the second degree is a violation.

\asection{Anti-Discrimination Clause}
This organization shall not discriminate on the basis of sex, race, color, sexual orientation, gender identity and gender expression, religion, age marital state, national origin, disability or veteran status.
This policy will include but is not limited to, recruiting, membership, organization activities or opportunities to hold or run for club office.

\asection{Statement of Compliance with University Regulations}
This organization shall comply with all university and Center for Campus Life policies and regulations, and local, state and federal laws.

\end{document}